\documentclass[a4paper,10pt]{article}
\usepackage[utf8]{inputenc}

%opening
\title{Redes de Ordenadores}
\author{}
\date{}

\begin{document}

\maketitle


\section{Memoria Descriptiva.}
\subsection{Datos generales.}
\subsubsection{Datos del promotor.}
\begin{itemize}
 \item Nombre:
 \item Domicilio:
 \item CIF:
 \item Población:
 \item Teléfono:
\end{itemize}

\subsubsection{Objeto del proyecto.}
El objetivo del proyecto es llevar a cabo la red de datos de la planta E2 y de los laboratorios 
de la Escuela Superior de Ingenieros de la Universidad de Sevilla. Además de la red de datos, 
se preveé la dotación de la red de VoIP, así como la de videovigilancia IP.
\subsubsection{Descripción del edificio o complejo urbano.}
El edificio sobre el cual se va a realizar la instalación se encuentra situado en el Camino de los
Descubrimientos s/n. La edificación en cuestión consta de planta baja más un sótano, junto con un
edificio de laboratorios que se encuentra, aproximadamente a 200m.

La planta baja consta de los departamentos de Telemática, Teoría de la Señal y Comunicaciones, 
Ingeniería Electrónica e Ingeniería de Sistemas y Automática. Además está equipada con una serie de
aulas y con el Centro de Cálculo, formando por despachos y aulas informáticas.

La función única del sótano es albergar los equipos que posibiliten la comuninación con el exterior
de la red objeto de este proyecto. El uso del edificio de laboratorios está destinado a los
departamentos de Telemática y de Teoría de la Señal y Telecomunicaciones.

En la siguiente tabla se resume la división que se hace de la ESI junto con los terminales que se pretenden instalar.

\begin{tabular}{|c|c|}
\hline   & \textbf{Terminales} \\ 
\hline  Telemática & 30 \\ 
\hline  Señales & 30 \\ 
\hline  Automática & 28 \\ 
\hline  Electrónica & 56 \\ 
\hline  VoIP & 199 \\ 
\hline  WIFI & 10 \\ 
\hline  Laboratorios Señal & 25 \\ 
\hline  Laboratorios Telemática & 30 \\ 
\hline  Servidores & 9 \\ 
\hline  Personal Admin y Servicios & 5 \\ 
\hline  CDC/despachos & 32 \\ 
\hline  CDC/aulas & 110 \\ 
\hline 
\end{tabular} \\ \\
Para el direccionamiento se preveen tanto direcciones privadas como públicas. Las IP públicas en el caso de los departamentos e IP privadas para el resto de la Escuela. En la tabla se resumen los rangos de direcciones a utilizar.\\

\begin{tabular}{|c|c|c|}
\hline \textbf{Direccionamiento público} & IP & Terminales \\ 
\hline Telemática & 193.147.160.0/27 & 30 \\ 
\hline Teoría de la Señal & 193.147.160.32/27 & 30 \\ 
\hline Automática & 193.147.160.64/27 & 28 \\ 
\hline Electrónica & 193.147.160.96/26  & 56 \\ 
\hline Servidores & 193.147.160.160/28 & 9 \\ 
\hline 
\end{tabular} \\ \\

En el caso de los servidores se tiene uno por cada departamento, dedicado a tareas de impresión e intercambio de archivos, y los que se encuentran en el Centro de Cálculo que son:
\begin{itemize}
	\item Servidor Web
	\item Base de datos (acceso exclusivo Servidor Web)
	\item Servidor Proxy
	\item Servidor VoIP
	\item Servidor VideoIP
\end{itemize}

\begin{tabular}{|c|c|c|}
\hline  \textbf{Direccionamiento Privado} & IP & Terminales \\ 
\hline  CDC/Despachos & 172.16.0.0/27 & 32 \\ 
\hline  CDC/Aulas & 172.16.0.32/25 & 110 \\ 
\hline  VoIP & 172.16.0.160/24 & 199 \\ 
\hline  VideoIP & 172.16.1.160/27 & 18 \\ 
\hline  Lab Teoría Señal & 172.16.1.192/27 & 25 \\ 
\hline  Aulas & 172.16.1.224/27 & 71 \\ 
\hline  Lab Telemática & 172.16.2.98/27 & 30 \\ 
\hline  Servidores & 172.16.2.130/28 & 9 \\ 
\hline  WIFI & 172.16.2.146/28 & 10 \\ 
\hline  Administración y Servicios & 172.16.2.l54/29 & 5 \\ 
\hline 
\end{tabular} 


\subsection{Dimensionamiento.}
\subsubsection{Enlace descendente}

\begin{tabular}{|c|c|c|c|c|}
\hline  & Ton & Toff & Tamaño (Mb) & Reg. Binario (Mbps) \\ 
\hline HTTP & 3 & 57 & 1.25 & 3.33 \\ 
\hline FTP & 15 & 285 & 4 & 2.13 \\ 
\hline POP3/SMTP & 2 & 58 & 0.3 & 1.2 \\ 
\hline  &  &  & Reg. Bin (Total) & 6.66 \\ 
\hline 
\end{tabular} 

\subsubsection{Enlace ascendente}

\begin{tabular}{|c|c|c|c|c|}
\hline  & Ton & Toff & Tamaño (Mb) & Reg. Binario (Mbps) \\ 
\hline HTTP & 3 & 57 & 0.02 & 0.053 \\ 
\hline FTP & 15 & 285 & 0.4 & 1.066 \\ 
\hline POP3/SMTP & 2 & 58 & 0.3 & 0.8 \\ 
\hline  &  &  & Reg. Bin (Total) & 1.92 \\ 
\hline 
\end{tabular} \\ \\
En el caso de las cámaras IP se estima un régimen binario de 4Mbps y para la telefonía IP, con la norma G.711, un régimen binario de 64Kbps.

\section{Planos.}
\section{Pliego de Condiciones.}
\subsection{Condiciones particulares.}
\subsection{Condiciones generales.}
\section{Presupuesto.}

\end{document}
