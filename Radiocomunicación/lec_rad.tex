\documentclass[twocolumn]{article}
\usepackage[spanish] {babel}
\usepackage[utf8] {inputenc}
\usepackage[T1] {fontenc}

%titulos y subtitulos
\title{Lecturas obligatorias de Radiación y Radiocomunicación}
\author{}
%eliminar fecha del encabezado
\date{}

%inicio
\begin{document}

\maketitle

\begin{abstract}

Pequeño resumen de cada una de las diez lecturas obligatorias de Radiación y Radiocomunicación

\end{abstract}



\section{Johan Ansdjö: Consejero Delegado de Yoigo}
Los tres pilares de la entrevista son la guerra de precios emprendida por las operadoras con red propia, la apuesta por la banda de 1800Mhz y la renuncia a la de los 900Mhz por representar un mal negocio para la compañia, y el despliegue de nuevas infraestructuras. También refleja que el Ingeniero de Telecomunicaciones español se centra más en la parte técnica, mientras que en Suecia la profesión se orienta hacia el cliente final; a vincular la solución a la funcionalidad de la misma, es decir, al uso que le dará el cliente.\\

Dentro de lo que es la figura del cliente hace referencia a que en España se es muy reacio a probar nuevos servicios, de hecho existe una penetración de conectividad USB muy baja. Los cambios en España en la comunicación móvil vendrán determinados por la crisis económica y por la incorporación de nuevos dispositivos. Los factores que influyen en el comportamiento del cliente son el precio, las aplicaciones, el terminal y los servicios de valor añadido.\\

En lo que respecta a Yoigo, que es una filial de Telia Sonera, su negocio se divide en un 80% de voz y datos y en un 20% de servicios de valor añadido. Sus líneas de funcionamiento son la simplificación de las tarifas, el uso de la banda de 1800Mhz para lanzar 4G, aumentar su red propia y actualizar sus instalaciones para el despliegue de LTE. La ventaja de LTE (1800Mhz) es que puede reutilizar las estaciones base UMTS.\\ 

Como curiosidades, prevee una demanda de capacidad infinita y cree que la única invención propia, o novedad, de la tecnología móvil es el SMS.

\section{}






\end{document}