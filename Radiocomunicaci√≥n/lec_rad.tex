\documentclass[twocolumn]{article}
\usepackage[spanish] {babel}
\usepackage[utf8] {inputenc}
\usepackage[T1] {fontenc}

%titulos y subtitulos
\title{Lecturas obligatorias de Radiación y Radiocomunicación}
\author{}
%eliminar fecha del encabezado
\date{}

%inicio
\begin{document}

\maketitle

\begin{abstract}

Pequeño resumen de cada una de las diez lecturas obligatorias de Radiación y Radiocomunicación

\end{abstract}



\section{Johan Ansdjö: Consejero Delegado de Yoigo}
Los tres pilares de la entrevista son la guerra de precios emprendida por las operadoras con red propia, la apuesta por la banda de 1800Mhz y la renuncia a la de los 900Mhz por representar un mal negocio para la compañia, y el despliegue de nuevas infraestructuras. También refleja que el Ingeniero de Telecomunicaciones español se centra más en la parte técnica, mientras que en Suecia la profesión se orienta hacia el cliente final; a vincular la solución a la funcionalidad de la misma, es decir, al uso que le dará el cliente.

Dentro de lo que es la figura del cliente hace referencia a que en España se es muy reacio a probar nuevos servicios, de hecho existe una penetración de conectividad USB muy baja. Los cambios en España en la comunicación móvil vendrán determinados por la crisis económica y por la incorporación de nuevos dispositivos. Los factores que influyen en el comportamiento del cliente son el precio, las aplicaciones, el terminal y los servicios de valor añadido.

En lo que respecta a Yoigo, que es una filial de Telia Sonera, su negocio se divide en un 80\% de voz y datos y en un 20\% de servicios de valor añadido. Sus líneas de funcionamiento son la simplificación de las tarifas, el uso de la banda de 1800Mhz para lanzar 4G, aumentar su red propia y actualizar sus instalaciones para el despliegue de LTE. La ventaja de LTE (1800Mhz) es que puede reutilizar las estaciones base UMTS.

Como curiosidades, prevee una demanda de capacidad infinita y cree que la única invención propia, o novedad, de la tecnología móvil es el SMS.

\section{El acceso a internet por satélite resiste el terremoto de Haití. La banda ancha inalámbrica enlaza las organizaciones de ayuda}
A raíz del terremoto, el IEEE Spectrum informó sobre el sistema de telefonía móvil de Haití, los radioaficionados y otras infraestructuras. Además el IEEE estableció un Fondo para la Reconstrucción de Haití.

El terremoto de magnitud 7.0 que azotó Haití dañó gravemente las telecomunicaciones, lo que dificultó el trabajo de los servicios de ayuda.

Se destruyeron casi todas las redes de comunicaciones, salvo el acceso a internet vía satélite; que fue fundamental para que el trabajo sobre el terreno de las ONG diera sus frutos.

La mayoría de los ISP de Haití, y por ende, de los países subdesarrollados son vía satélite, por lo que no se dependía del cable submarino de 1.92Tbps que fue destruido, y tardará tiempo en ser reparado. Esa tecnología satélite conocida como VSAT (terminal de muy pequeña abertura), conecta las estaciones base terrenales con las naves espaciales en órbitas geosíncronas lejanas.

"VSAT es generalmente la manera más barata y rápida de mantener a un país conectado a Internet", según Hernán Galperín, profesor de la telecomunicaciones de la Universidad de San Andrés, en Buenos Aires. La ventaja de VSAT es que tiende hacía una red distribuida, como la propia internet.

NetHope, una "colaboración de las 28 principales organizaciones humanitairas internacionales", fue una de las ONG que se aprovecharon del servicio de satélite. Está organización está trabajando con Inveneo, con sede en San Francisco, para establecer la conexión a Internet en la capital haitiana de Port-au-Prince y sus alrededores a través de enlaces VSAT combinados con Wi-Fi de largo alcance. La red Inveneo admite el acceso a Internet dentro y fuera del país, lleva las comunicaciones de voz, y permite la colaboración y el intercambio de recursos entre un máximo de 20 organizaciones no gubernamentales.

El cofundador de Inveneo Mark Summer y el ingeniero Andris Bjornson llegaron a Haití con más de una decena de antenas parabólicas RocketDish de 5Ghz y 10 MiniServers Linux, entre otro mucho equipo. Para restablecer la conexión conectaron un servidor a un satélite VSAT mediante un enlace descendente desde ITC Global, e instalaron un punto de acceso local en la sede de CHF International. A partir de ahí se crearon dos enlaces Wi-Fi largos que conectaban dos oficinas diferentes de Save The Children Federation en Port-au-Prince. Además esa misma tarde se estableció otro enlace a las oficinas de Catholic Relief Services.

Debido a que la telefonía móvil tenía en Haití una penetración bastante alta antes del terremoto, es predecible que la mayoría de la población tendrá acceso a través de los dispositivos móviles. James Cowie, director de tecnología de Renesys comentó que "Me gustaría pensar que el acceso barato a Internet móvil podría llegar a ser uno de los factores pequeños que mejorarán la educación y crearán oportunidades económicas para los haitianos en el largo plazo", dice. "Pero ante tanto sufrimiento, obviamente tienen preocupaciones mucho más graves e inmediatas para hacer frente este año".





\end{document}